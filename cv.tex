%-----------------------------------------------------------------------------------------------------------------------------------------------%
%	The MIT License (MIT)
%
%	Copyright (c) 2021 Jitin Nair
%
%	Permission is hereby granted, free of charge, to any person obtaining a copy
%	of this software and associated documentation files (the "Software"), to deal
%	in the Software without restriction, including without limitation the rights
%	to use, copy, modify, merge, publish, distribute, sublicense, and/or sell
%	copies of the Software, and to permit persons to whom the Software is
%	furnished to do so, subject to the following conditions:
%	
%	THE SOFTWARE IS PROVIDED "AS IS", WITHOUT WARRANTY OF ANY KIND, EXPRESS OR
%	IMPLIED, INCLUDING BUT NOT LIMITED TO THE WARRANTIES OF MERCHANTABILITY,
%	FITNESS FOR A PARTICULAR PURPOSE AND NONINFRINGEMENT. IN NO EVENT SHALL THE
%	AUTHORS OR COPYRIGHT HOLDERS BE LIABLE FOR ANY CLAIM, DAMAGES OR OTHER
%	LIABILITY, WHETHER IN AN ACTION OF CONTRACT, TORT OR OTHERWISE, ARISING FROM,
%	OUT OF OR IN CONNECTION WITH THE SOFTWARE OR THE USE OR OTHER DEALINGS IN
%	THE SOFTWARE.
%	
%
%-----------------------------------------------------------------------------------------------------------------------------------------------%

%----------------------------------------------------------------------------------------
%	DOCUMENT DEFINITION
%----------------------------------------------------------------------------------------

% article class because we want to fully customize the page and not use a cv template
\documentclass[a4paper,12pt]{article}

%----------------------------------------------------------------------------------------
%	FONT
%----------------------------------------------------------------------------------------

% % fontspec allows you to use TTF/OTF fonts directly
% \usepackage{fontspec}
% \defaultfontfeatures{Ligatures=TeX}

% % modified for ShareLaTeX use
% \setmainfont[
% SmallCapsFont = Fontin-SmallCaps.otf,
% BoldFont = Fontin-Bold.otf,
% ItalicFont = Fontin-Italic.otf
% ]
% {Fontin.otf}

%----------------------------------------------------------------------------------------
%	PACKAGES
%----------------------------------------------------------------------------------------
\usepackage{url}
\usepackage{parskip} 	

%other packages for formatting
\RequirePackage{color}
\RequirePackage{graphicx}
\usepackage[usenames,dvipsnames]{xcolor}
\usepackage[scale=0.9]{geometry}

%tabularx environment
\usepackage{tabularx}

%for lists within experience section
\usepackage{enumitem}
\setlist[itemize]{noitemsep}

% centered version of 'X' col. type
\newcolumntype{C}{>{\centering\arraybackslash}X} 

%to prevent spillover of tabular into next pages
\usepackage{supertabular}
\usepackage{tabularx}
\newlength{\fullcollw}
\setlength{\fullcollw}{0.47\textwidth}

%custom \section
\usepackage{titlesec}				
\usepackage{multicol}
\usepackage{multirow}

%CV Sections inspired by: 
%http://stefano.italians.nl/archives/26
\titleformat{\section}{\Large\scshape\raggedright}{}{0em}{}[\titlerule]
\titlespacing{\section}{0pt}{10pt}{10pt}

%for publications
\usepackage[style=chem-acs,sorting=ydnt]{biblatex}

%Setup hyperref package, and colours for links
\usepackage[unicode, draft=false]{hyperref}
\definecolor{linkcolour}{rgb}{0,0.2,0.6}
\hypersetup{colorlinks,breaklinks,urlcolor=linkcolour,linkcolor=linkcolour}
\addbibresource{citations.bib}
\setlength\bibitemsep{1em}

%for social icons
\usepackage{fontawesome5}

%debug page outer frames
%\usepackage{showframe}

\setlength{\multicolsep}{1.0pt}% 50% of original values


% job listing environments
\newenvironment{joblocation}[2]
    {
    \begin{tabularx}{\linewidth}{@{}l X r@{}}
    \textit{#1} & \hfill &  #2 \\[3.75pt]
    \end{tabularx}
    }
    {
    }

\newenvironment{jobtitle}[2]
    {
    \vspace{4pt}
    \begin{tabularx}{\linewidth}{@{}l X r@{}}
    \textbf{#1} & \hfill &  #2 \\[3.75pt]
    \end{tabularx}
    \begin{minipage}[t]{\linewidth}
    \begin{itemize}[nosep, after=\strut, leftmargin=1em, itemsep=3pt, label=--]
    }
    {
    \end{itemize}
    \vspace{8pt}
    \end{minipage}
    }

\newenvironment{projecttitle}[1]
    {
    \begin{tabularx}{\linewidth}{@{}l X r@{}}
    \textbf{#1} & \hfill \\[3.75pt]
    \end{tabularx}
    }
    {
    }

\newenvironment{projectdescription}[0]
    {
    \vspace{4pt}
    \begin{minipage}[t]{\linewidth}
    \begin{itemize}[nosep, after=\strut, leftmargin=1em, itemsep=3pt]
    }
    {
    \end{itemize}
    \vspace{-8pt}
    \end{minipage}
    }



%----------------------------------------------------------------------------------------
%	BEGIN DOCUMENT
%----------------------------------------------------------------------------------------
\begin{document}

% non-numbered pages
\pagestyle{empty} 

%----------------------------------------------------------------------------------------
%	TITLE
%----------------------------------------------------------------------------------------

% \begin{tabularx}{\linewidth}{ @{}X X@{} }
% \huge{Your Name}\vspace{2pt} & \hfill \emoji{incoming-envelope} email@email.com \\
% \raisebox{-0.05\height}\faGithub\ username \ | \
% \raisebox{-0.00\height}\faLinkedin\ username \ | \ \raisebox{-0.05\height}\faGlobe \ mysite.com  & \hfill \emoji{calling} number
% \end{tabularx}

\begin{tabularx}{\linewidth}{@{} C @{}}
\Huge{\textbf{Ryne C. Johnston}} \\[7.5pt]
\href{https://github.com/rynecjohnston}{\raisebox{-0.05\height}\faGithub\ rynecjohnston} \ $|$ \ 
\href{https://linkedin.com/in/rynecjohnston}{\raisebox{-0.05\height}\faLinkedin\ rynecjohnston} \ $|$ \ 
\href{mailto:ryne.c.johnston@gmail.com}{\raisebox{-0.05\height}\faEnvelope \ ryne.c.johnston@gmail.com} \ $|$ \ 
\href{tel:+18704032201}{\raisebox{-0.05\height}\faMobile \ +1 (870) 403-2201} \\
\end{tabularx}

%----------------------------------------------------------------------------------------
% EXPERIENCE SECTIONS
%----------------------------------------------------------------------------------------

%Experience
\section{Professional Experience}

\begin{joblocation}{Schr\"odinger, Inc.}{Portland, OR}
\end{joblocation}
\begin{jobtitle}{Principal Scientist II/Product Manager}{Jun 2023 -- Present}
    \item Managed multiple products and directed teams to meet project goals and deadlines.
    \vspace{2pt}
    \begin{itemize}
        \item \textit{Epik}: p\textit{K}\textsubscript{a} and protonation state predictor
        \item \textit{E-sol}: Central nervous system permeability predictor
    \end{itemize}
    \item Interacted with customers and provided software solutions to meet their needs.
    \item Interacted with marketing and sales teams to develop and execute product launch plans.
\end{jobtitle}
\begin{jobtitle}{Principal Scientist I/Technical Lead}{Jun 2021 -- Jun 2023}
    \item Developed and maintained best-in-class Python software to predict chemical properties through machine learning.
    \item Contributed to the development of a Python workflow for the rational design of homogenous catalysts to unlock previously unattainable reactivity.
    \item Wrote scholarly articles and technical documents for public consumption.
    \item Wrote and maintained documentation and automated tests.
    \item Performed code reviews.
\end{jobtitle}
\begin{jobtitle}{Senior Scientist}{May 2017 -- Jun 2021}
    \item Developed software to enumerate and canonicalize different related chemical compounds.
    \item Maintained legacy chemistry C/C++ software.
\end{jobtitle}
\begin{joblocation}{Oak Ridge National Laboratory}{Oak Ridge, TN}
\end{joblocation}
\begin{jobtitle}{Postdoctoral Scholar}{Jun 2015 -- Feb 2017}
    \item Used quantum chemistry software to investigate environmentally relevant biogeochemical processes.
    \item Probed the mechanisms of how inorganic mercury is microbially assimilated into a bioaccumulative organic form and its later environmentally-mediated deposition back into its inorganic form.
    \item Developed methods for computing p\textit{K}\textsubscript{a} values of environmentally relevant functional groups and aqueous formation constants of metal complexes.
    \item Wrote scholarly articles and research project proposals.
    \item Awarded nearly \$1M in research grant funds.
\end{jobtitle}
  
%Projects
\section{Projects}

\begin{projecttitle}{Epik}
\end{projecttitle}
\begin{projectdescription}
\item I designed, developed, and now manage \href{https://www.schrodinger.com/platform/products/epik/}{\textit{Epik}}, a Python software program for predicting the aqueous p\textit{K}\textsubscript{a} values and distribution of protonation states of small molecules. It leverages a machine learning model trained on 77k p\textit{K}\textsubscript{a} values for accurate predictions across broad chemical space.  \\
\end{projectdescription}
\begin{projecttitle}{E-sol}
\end{projecttitle}
\begin{projectdescription}
\item I am the product manager of \href{https://pubs.acs.org/doi/abs/10.1021/acs.jcim.3c00150}{\textit{E-sol}}, a workflow using physics-based methods for calculating partition coefficients of small molecules with the aim of predicting their central nervous system permeability.  \\
\end{projectdescription}
\begin{projecttitle}{PetroSim}
\end{projecttitle}
\begin{projectdescription}
\item As a passion side project, I created the open-source \href{https://github.com/rynecjohnston/petrosim}{\textit{PetroSim}} Python package as a collection of Python ports of the originally Visual Basic EC-(R)AFC petrological models for simulating the evolution of the geochemical composition of a magma body over time.  \\
\end{projectdescription}

%----------------------------------------------------------------------------------------
%	EDUCATION
%----------------------------------------------------------------------------------------
\section{Education}

\begin{tabularx}{\textwidth} {l l r}   
    2015 & \textbf{Doctor of Philosophy}, Computational Chemistry & \textit{Oregon State University} \\
    2010 & \textbf{Bachelor of Science}, Chemistry & \textit{Henderson State University} \\
\end{tabularx}
\vspace{6pt}

%----------------------------------------------------------------------------------------
%	PUBLICATIONS
%----------------------------------------------------------------------------------------
\section{Publications}
\begin{refsection}[works.bib]
\nocite{*}
\printbibliography[heading=none]
\end{refsection}

%----------------------------------------------------------------------------------------
%	SKILLS
%----------------------------------------------------------------------------------------
\section{Skills}
\begin{multicols}{3}
    \begin{itemize}
        \item Python
        \item NumPy
        \item scikit-learn
        \item pandas
        \item PyTorch
        \item RDKit
        \item version control (git/GitHub)
        \item C++
        \item Fortran
        \item agile methodology
        \item productivity tools
        \item performance profilers
        \item computational chemistry packages
        \begin{itemize}
            \item Schr\"odinger
            \item ORCA
        \end{itemize}
    \end{itemize}
\end{multicols}

\section{Interests}
\begin{itemize}[nosep, after=\strut, leftmargin=1em, itemsep=3pt]
\item chemistry, geology, history, programming, hiking
\end{itemize}

\vfill
\center{\footnotesize Last updated: \today}

\end{document}
